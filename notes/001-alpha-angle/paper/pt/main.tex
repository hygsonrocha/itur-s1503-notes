% !TeX program = lualatex
% !TeX TXS-program:compile = txs:///lualatex -synctex=1 -interaction=nonstopmode

\documentclass[12pt,a4paper]{article}

% ----------------------------
% 0. Motor / robustez
% ----------------------------
\usepackage{iftex}
\ifPDFTeX
\errmessage{Este modelo deve ser compilado com LuaLaTeX (ou XeLaTeX).}
\fi

\hfuzz=6pt
\vfuzz=6pt

% ----------------------------
% 1. Layout e tipografia
% ----------------------------
\usepackage[a4paper,margin=2.5cm]{geometry}
\usepackage{microtype}
\usepackage[brazilian,shorthands=off]{babel}

\setlength{\parindent}{0pt}
\setlength{\parskip}{0.35em}

\setlength{\abovedisplayskip}{0.35em}
\setlength{\belowdisplayskip}{0.35em}
\setlength{\abovedisplayshortskip}{0.20em}
\setlength{\belowdisplayshortskip}{0.20em}

% ----------------------------
% 2. Fontes (LuaLaTeX)
% ----------------------------
\usepackage{fontspec}
\defaultfontfeatures{Ligatures=TeX, Scale=MatchLowercase}

\usepackage{mathtools}
\usepackage{unicode-math}

\IfFontExistsTF{STIX Two Text}{
	\setmainfont{STIX Two Text}
}{
	\IfFontExistsTF{Libertinus Serif}{
		\setmainfont{Libertinus Serif}
	}{
		\setmainfont{Latin Modern Roman}
	}
}

\IfFontExistsTF{STIX Two Math}{
	\setmathfont{STIX Two Math}
}{
	\IfFontExistsTF{Libertinus Math}{
		\setmathfont{Libertinus Math}
	}{
		\setmathfont{Latin Modern Math}
	}
}

\IfFontExistsTF{Libertinus Sans}{
	\setsansfont{Libertinus Sans}
}{
	\setsansfont{TeX Gyre Heros}
}

\IfFontExistsTF{Inconsolata}{
	\setmonofont{Inconsolata}[Scale=MatchLowercase]
}{
	\IfFontExistsTF{Latin Modern Mono}{
		\setmonofont{Latin Modern Mono}[Scale=MatchLowercase]
	}{
		\setmonofont{Libertinus Mono}[Scale=MatchLowercase]
	}
}

% ----------------------------
% 3. Cores e caixas
% ----------------------------
\usepackage{xcolor}
\definecolor{AnatelBlue}{HTML}{004B87}
\definecolor{AnatelLight}{HTML}{F0F5FA}

\usepackage{tcolorbox}
\tcbuselibrary{skins,breakable}

\newtcolorbox{block}[1]{
	breakable, enhanced,
	colback=AnatelBlue!5,
	colframe=AnatelBlue!35,
	boxrule=0.6pt,
	arc=2pt,
	left=6pt,right=6pt,top=6pt,bottom=6pt,
	title=\textbf{#1},
	coltitle=AnatelBlue,
	fonttitle=\normalsize,
}

\newtcolorbox{alertblock}[1]{
	breakable, enhanced,
	colback=AnatelLight,
	colframe=AnatelBlue,
	boxrule=0.9pt,
	arc=2pt,
	left=6pt,right=6pt,top=6pt,bottom=6pt,
	title=\textbf{#1},
	colbacktitle=AnatelBlue,
	coltitle=white,
	fonttitle=\normalsize,
}

% ----------------------------
% 4. Pacotes úteis
% ----------------------------
\usepackage{siunitx}
\usepackage{booktabs}
\usepackage{enumitem}

\sisetup{
	mode = match,
	propagate-math-font = true,
	reset-math-version = false,
	reset-text-family = false,
	reset-text-series = false,
	reset-text-shape = false,
	text-family-to-math = true,
	text-series-to-math = true,
	output-decimal-marker = {,},
	inter-unit-product = \ensuremath{{}\cdot{}}
}

\newcommand{\tightlist}{%
	\setlist[itemize]{itemsep=0.25em,topsep=0.25em,parsep=0pt,partopsep=0pt}%
	\setlist[enumerate]{itemsep=0.25em,topsep=0.25em,parsep=0pt,partopsep=0pt}%
}

% ----------------------------
% 5. Figuras (TikZ)
% ----------------------------
\usepackage{tikz}
\usetikzlibrary{calc}

% ----------------------------
% 6. Links
% ----------------------------
\usepackage{hyperref}
\hypersetup{
	colorlinks=true,
	linkcolor=AnatelBlue,
	urlcolor=AnatelBlue,
	citecolor=AnatelBlue,
	pdftitle={Cálculo do ângulo alfa (ITU-R S.1503-4)},
	pdfauthor={Hygson A. P. da Rocha}
}
\urlstyle{same}

% ----------------------------
% 7. Macros
% ----------------------------
\newcommand{\Rgeo}{\ensuremath{R_{\mathrm{geo}}}}
\newcommand{\Rearth}{\ensuremath{R_{e}}}
\newcommand{\Lon}{\ensuremath{\mathrm{Long}}}
\newcommand{\wrapPi}[1]{\operatorname{wrap}_{\pi}\!\left(#1\right)}

\title{Nota 001 -- Cálculo do ângulo \texorpdfstring{$\alpha$}{alpha} ao arco GSO visível\\(referência: ITU-R S.1503-4)}
\author{Hygson A. P. da Rocha}
\date{\today}

% Sumário opcional (padrão: sem)
\newif\ifwithtoc
\withtocfalse   % <-- sem sumário (padrão)
% \withtoctrue  % <-- com sumário

\begin{document}
	\maketitle
	
	\begin{abstract}
		Esta nota consolida (i) a definição geométrica do ângulo \(\alpha\) da Recomendação ITU-R S.1503-4,
		(ii) a caracterização do arco GSO visível a partir de uma estação terrena (ES) no modelo esférico,
		e (iii) uma estratégia prática para obter \(|\alpha|\) e \(\Lon_{\alpha}\) (longitude do ponto do arco que minimiza \(\alpha\)),
		incluindo o critério de sinal da própria Recomendação.
	\end{abstract}
	
	\ifwithtoc
	\tableofcontents
	\medskip
	\fi
	
	% ============================================================
	\section*{Escopo e reprodutibilidade}
	\addcontentsline{toc}{section}{Escopo e reprodutibilidade}
	
	Este documento foi pensado para acompanhar uma implementação reprodutível:
	\begin{itemize}\tightlist
		\item Código Python: \texttt{src/alpha\_angle.py}
		\item Notebook de demonstração: \texttt{notebooks/alpha\_angle\_demo.ipynb}
	\end{itemize}
	
	A ideia é manter esta nota \emph{sem} a colagem integral de código-fonte (para reduzir redundância, evitar divergências e simplificar a compilação em \LaTeX).
	A implementação de referência do algoritmo está no arquivo \texttt{src/alpha\_angle.py}, enquanto o notebook serve como roteiro executável, com exemplos de uso e checagens numéricas.
	
	\begin{alertblock}{Compilação}
		Este \texttt{main.tex} compila com Lua\LaTeX. Ele não depende de \texttt{minted}/Pygments, pois não embute código.
	\end{alertblock}
	
	% ============================================================
	\section*{Definição de \texorpdfstring{$\alpha$}{alpha} e notação}
	\addcontentsline{toc}{section}{Definição de alpha e notação}
	
	\paragraph{Definição (módulo).}
	O ângulo \(\alpha\) é a separação angular instantânea, medida no receptor (ES), entre a linha de visada ES\(\rightarrow\)NGSO
	e a família de linhas de visada ES\(\rightarrow\)ponto no arco GSO visível. Em termos do módulo,
	\[
	|\alpha| \;=\; \min_{\lambda\in\mathcal I}\ \arccos\!\big(\mathbf u_N\cdot \mathbf u_G(\lambda)\big)
	\;=\; \arccos\!\Big(\max_{\lambda\in\mathcal I}\ \mathbf u_N\cdot \mathbf u_G(\lambda)\Big),
	\]
	onde \(\mathcal I\) é o intervalo de longitudes do arco GSO \emph{visível} a partir da estação terrena.
	
	\paragraph{Definição (sinal) e longitudes associadas.}
	Conforme a S.1503, \(\alpha\) é \emph{assinado}. Após determinar \(|\alpha|\) e o ponto do arco GSO que o minimiza,
	defina:
	\[
	\Lon_{\alpha}\equiv \Lon_{\mathrm{GSO}}(\lambda^\star)=\lambda^\star,
	\qquad
	\Delta\Lon=\wrapPi{\Lon_{\alpha}-\Lon_{\mathrm{NGSO}}},
	\]
	onde \(\Lon_{\mathrm{NGSO}}\) é a longitude geocêntrica do satélite NGSO. O sinal de \(\alpha\) é obtido por um teste geométrico
	com base na interseção da reta ES\(\rightarrow\)NGSO com o plano equatorial \(XY\) (Seção \ref{sec:sinal}).
	
	\paragraph{Notação (vetores).}
	Sejam:
	\begin{itemize}\tightlist
		\item \(\mathbf r_E\in\mathbb R^3\): posição geocêntrica (ECEF) da estação terrena GSO vítima;
		\item \(\mathbf r_N\in\mathbb R^3\): posição geocêntrica (ECEF) do satélite NGSO;
		\item \(\mathbf r_G(\lambda)\in\mathbb R^3\): ponto no anel GEO (círculo equatorial de raio \(\Rgeo\)) na longitude geocêntrica \(\lambda\):
		\[
		\mathbf{r}_G(\lambda) \;=\; \Rgeo
		\begin{bmatrix}\cos\lambda\\ \sin\lambda\\ 0\end{bmatrix}.
		\]
	\end{itemize}
	
	Defina os vetores unitários de visada:
	\[
	\mathbf u_N=\frac{\mathbf r_N-\mathbf r_E}{\|\mathbf r_N-\mathbf r_E\|},
	\qquad
	\mathbf u_G(\lambda)=\frac{\mathbf r_G(\lambda)-\mathbf r_E}{\|\mathbf r_G(\lambda)-\mathbf r_E\|}.
	\]
	Para uma longitude \(\lambda\) no anel GEO, a separação é
	\[
	\alpha(\lambda)=\arccos\!\big(\mathbf u_N\cdot \mathbf u_G(\lambda)\big),
	\qquad
	|\alpha|=\min_{\lambda\in\mathcal I}\alpha(\lambda).
	\]
	
	% ============================================================
	\section*{Arco GSO visível a partir da ES}
	\addcontentsline{toc}{section}{Arco GSO visível a partir da ES}
	
	\paragraph{Latitude/longitude geocêntricas da ES.}
	Seja \(\varphi_E\) a latitude geocêntrica da ES e \(\lambda_E\) sua longitude geocêntrica:
	\[
	\varphi_E=\arcsin\!\left(\frac{z_E}{\|\mathbf r_E\|}\right),
	\qquad
	\lambda_E=\operatorname{atan2}(y_E,x_E).
	\]
	
	\paragraph{Ideia física (horizonte local).}
	No modelo esférico, a direção radial \(\mathbf r_E\) é a normal do plano tangente (horizonte local) na estação.
	Um ponto \(\mathbf r\) no espaço está \emph{acima do horizonte} (elevação \(\ge 0\)) se o vetor linha-de-visada
	\((\mathbf r-\mathbf r_E)\) forma ângulo agudo (ou reto) com a normal \(\mathbf r_E\), i.e.:
	\[
	(\mathbf r-\mathbf r_E)\cdot \mathbf r_E \;\ge\; 0
	\quad\Longleftrightarrow\quad
	\mathbf r\cdot \mathbf r_E \;\ge\; \|\mathbf r_E\|^2.
	\]
	
	\paragraph{Condição de visibilidade no anel GEO.}
	Para um ponto no anel GEO, \(\mathbf r=\mathbf r_G(\lambda)\), a visibilidade equivale a:
	\[
	\mathbf r_G(\lambda)\cdot \mathbf r_E \;\ge\; \|\mathbf r_E\|^2.
	\]
	Escrevendo \(\mathbf r_E\) em termos de \(\varphi_E,\lambda_E\):
	\[
	\mathbf r_E= \|\mathbf r_E\|
	\begin{bmatrix}
		\cos\varphi_E\cos\lambda_E\\
		\cos\varphi_E\sin\lambda_E\\
		\sin\varphi_E
	\end{bmatrix},
	\quad\text{e}\quad
	\mathbf r_G(\lambda)=\Rgeo
	\begin{bmatrix}
		\cos\lambda\\
		\sin\lambda\\
		0
	\end{bmatrix},
	\]
	obtém-se:
	\[
	\mathbf r_G(\lambda)\cdot \mathbf r_E
	=
	\Rgeo\|\mathbf r_E\|\cos\varphi_E\cos(\lambda-\lambda_E).
	\]
	Portanto,
	\[
	\cos(\lambda-\lambda_E)\;\ge\;\frac{\|\mathbf r_E\|}{\Rgeo\cos\varphi_E}.
	\]
	
	\paragraph{Semi-extensão \texorpdfstring{$\theta_{\max}$}{theta\_max} e intervalo \(\mathcal I\).}
	A condição de visibilidade acima define as \emph{longitudes do anel GEO que estão acima do horizonte} da ES.
	No uso usual do modelo esférico, toma-se \(\|\mathbf r_E\|\approx \Rearth\). Assim:
	\[
	\cos(\lambda-\lambda_E)\;\ge\;\frac{\Rearth}{\Rgeo\cos\varphi_E}.
	\]
	Essa desigualdade define um intervalo simétrico em torno de \(\lambda_E\):
	\[
	\mathcal I=\bigl[\lambda_E-\theta_{\max},\,\lambda_E+\theta_{\max}\bigr],
	\qquad
	\theta_{\max}=\arccos\!\left(\frac{\Rearth}{\Rgeo\cos\varphi_E}\right),
	\]
	desde que \(\cos\varphi_E>\Rearth/\Rgeo\). A minimização de \(|\alpha|\) (equivalentemente, a maximização de \(f(\lambda)\))
	é realizada \emph{somente} para \(\lambda\in\mathcal I\); fora desse intervalo, os pontos do anel GEO não são visíveis a partir da ES.
	
	\paragraph{Caso sem arco visível.}
	Se \(\cos\varphi_E\le \Rearth/\Rgeo\), não há solução para a condição de visibilidade (nenhum ponto do anel GEO está acima do horizonte da ES).
	Nesse caso, adota-se a convenção operacional \(|\alpha|=\pi\).
	
	\begin{figure}[ht]
		\centering
		% (a) Vista superior no plano equatorial XY
		\begin{minipage}[t]{0.54\textwidth}
			\centering
			\begin{tikzpicture}[scale=1.05, line cap=round, line join=round]
				\def\ReD{1.35}
				\def\RgeoD{3.10}
				\def\rhoD{0.95}
				\def\thmaxD{55}
				\def\lamED{0}
				
				\pgfmathsetmacro{\RgeoLong}{\RgeoD*1.08}
				\pgfmathsetmacro{\RgeoShort}{\RgeoD*1.03}
				
				\draw[thick] (0,0) circle (\ReD);
				\draw[thick] (0,0) circle (\RgeoD);
				
				\fill (0,0) circle (1.2pt) node[below left] {$O$};
				
				\draw[->,thick] (0,0) -- (\lamED:\RgeoLong) node[above right] {$\lambda_E$};
				
				\draw[thick] (0,0) -- (\lamED+\thmaxD:\RgeoShort);
				\draw[thick] (0,0) -- (\lamED-\thmaxD:\RgeoShort);
				
				\draw[line width=2.2pt] (\lamED-\thmaxD:\RgeoD) arc (\lamED-\thmaxD:\lamED+\thmaxD:\RgeoD);
				
				\fill (\lamED+\thmaxD:\RgeoD) circle(1.2pt)
				node[above] {$\lambda_E+\theta_{\max}$};
				\fill (\lamED-\thmaxD:\RgeoD) circle(1.2pt)
				node[below] {$\lambda_E-\theta_{\max}$};
				
				\fill (\lamED:\rhoD) circle (1.2pt) node[below] {$E_\perp$};
				\draw[dashed] (0,0) -- (\lamED:\rhoD)
				node[pos=0.65, below=10pt, xshift=6pt] {\small $\rho_E=\Rearth\cos\varphi_E$};
				
				\draw[->] (\lamED:0.80) arc (\lamED:\lamED+\thmaxD:0.80);
				\node at (\lamED+\thmaxD/2:1.05) {\small $\theta_{\max}$};
				
				\node at (0,-\RgeoD-0.55) {\small (a) Plano equatorial \(XY\): arco visível no anel GEO};
			\end{tikzpicture}
		\end{minipage}
		\hfill
		% (b) Horizonte local
		\begin{minipage}[t]{0.43\textwidth}
			\centering
			\begin{tikzpicture}[scale=1.05, line cap=round, line join=round]
				\def\ReD{1.35}
				\def\phiD{35}
				
				\coordinate (O) at (0,0);
				\coordinate (E) at ({\ReD*cos(\phiD)},{\ReD*sin(\phiD)});
				
				\draw[thick] (O) circle (\ReD);
				\fill (O) circle (1.2pt) node[below left] {$O$};
				
				\draw[thick,->] (O) -- (E) node[midway,above left] {$\mathbf r_E$};
				\fill (E) circle (1.2pt) node[above right] {$E$};
				
				\pgfmathsetmacro{\tx}{-sin(\phiD)}
				\pgfmathsetmacro{\ty}{cos(\phiD)}
				
				\coordinate (T1) at ($(E)+({0.95*\tx},{0.95*\ty})$);
				\coordinate (T2) at ($(E)-({0.95*\tx},{0.95*\ty})$);
				\draw[thick] (T1) -- (T2);
				\node[above] at ($(E)+(-0.5,0.65)$) {\small plano tangente (horizonte)};
				
				\coordinate (G) at ($(E)+(1.65,0.55)$);
				\fill (G) circle (1.2pt) node[above right] {$\mathbf r_G(\lambda)$};
				\draw[->] (E) -- (G) node[midway,above] {$\mathbf r_G-\mathbf r_E$};
				
				\node at (0,-\ReD-0.55) {\small (b) Visibilidade: \((\mathbf r_G-\mathbf r_E)\cdot\mathbf r_E \ge 0\)};
			\end{tikzpicture}
		\end{minipage}
		
		\caption{Interpretação geométrica do arco GEO visível a partir da ES no modelo esférico.}
		\label{fig:geo-visible-arc}
	\end{figure}
	
	% ============================================================
	\section*{Minimização de \texorpdfstring{$\alpha(\lambda)$}{alpha(lambda)} como maximização}
	\addcontentsline{toc}{section}{Minimizacao de alpha(lambda) como maximizacao}
	
	Minimizar \(\alpha(\lambda)\) equivale a maximizar o produto escalar:
	\[
	f(\lambda)=\mathbf u_N\cdot \mathbf u_G(\lambda)
	=
	\frac{(\mathbf r_N-\mathbf r_E)\cdot(\mathbf r_G(\lambda)-\mathbf r_E)}
	{\|\mathbf r_N-\mathbf r_E\|\,\|\mathbf r_G(\lambda)-\mathbf r_E\|}.
	\]
	A implementação em \texttt{src/alpha\_angle.py} usa uma formulação analítica no frame rotacionado por \(-\lambda_E\),
	gerando um conjunto finito de candidatos (bordas do arco, ponto central e raízes de uma quartica associada à condição de estacionariedade),
	seguido de desempate por \(\Delta\Lon\) quando houver empates numéricos.
	
	\begin{block}{Regra de desempate por \(\Delta\Lon\)}
		Se existirem múltiplos \(\lambda^\star\) que produzam o mesmo \(|\alpha|\) (ou indistinguíveis dentro de uma tolerância numérica),
		escolha aquele que minimiza \(|\Delta\Lon|\). Persistindo empate, escolha \(\Delta\Lon>0\).
	\end{block}
	
	% ============================================================
	\section*{Critério de sinal de \texorpdfstring{$\alpha$}{alpha}}
	\addcontentsline{toc}{section}{Criterio de sinal de alpha}
	\label{sec:sinal}
	
	O critério operacional (S.1503, Anexo/Apêndice D, \S D6.4.4.3) pode ser implementado assim:
	
	\begin{enumerate}\tightlist
		\item Considere a reta \(P(t)=\mathbf r_E+t(\mathbf r_N-\mathbf r_E)\).
		\item Interseção com \(z=0\): se \((r_{N,z}-r_{E,z})\neq 0\), então \(t_0=-r_{E,z}/(r_{N,z}-r_{E,z})\).
		\item Se \(t_0<0\), a interseção está ``atrás'' da ES; trate como \(\rho=\infty\).
		\item Caso contrário, compute \(P(t_0)\) e \(\rho=\sqrt{P_x(t_0)^2+P_y(t_0)^2}\).
		\item Então:
		\begin{itemize}\tightlist
			\item se \(r_{E,z}>0\) (hemisfério norte), \(\alpha>0\) se \(\rho<\Rgeo\), caso contrário \(\alpha<0\);
			\item se \(r_{E,z}<0\) (hemisfério sul), \(\alpha>0\) se \(\rho>\Rgeo\), caso contrário \(\alpha<0\).
		\end{itemize}
	\end{enumerate}
	
	\begin{alertblock}{Observação (casos degenerados)}
		A rotina de referência inclui tratamentos numéricos para casos degenerados (por exemplo, ES no equador e/ou visada quase paralela ao plano \(XY\)).
	\end{alertblock}
	
	% ============================================================
	\section*{Como usar o código}
	\addcontentsline{toc}{section}{Como usar o codigo}
	
	\paragraph{Arquivos.}
	No repositório, mantenha:
	\begin{itemize}\tightlist
		\item \texttt{src/alpha\_angle.py}: implementação de referência do algoritmo;
		\item \texttt{notebooks/alpha\_angle\_demo.ipynb}: exemplos de uso, validações e reprodução dos resultados.
	\end{itemize}
	
	\begin{alertblock}{Constantes esféricas}
		Defina e documente explicitamente os valores de \(\Rearth\) e \(\Rgeo\) adotados na implementação.
		Na prática, ambos aparecem na literatura/implementações: \(\Rearth=\SI{6371}{km}\) ou \(\SI{6378.145}{km}\);
		\(\Rgeo\approx \SI{42164}{km}\).
	\end{alertblock}
	
\end{document}
